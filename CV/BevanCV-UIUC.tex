%%%%%%%%%%%%%%%%%%%%%%%%%%%%%%%%%%%%%%%%%
% Medium Length Graduate Curriculum Vitae
% LaTeX Template
% Version 1.1 (9/12/12)
%
% This template has been downloaded from:
% http://www.LaTeXTemplates.com
%
% Original author:
% Rensselaer Polytechnic Institute (http://www.rpi.edu/dept/arc/training/latex/resumes/)
%
% Important note:
% This template requires the res.cls file to be in the same directory as the
% .tex file. The res.cls file provides the resume style used for structuring the
% document.
%
%%%%%%%%%%%%%%%%%%%%%%%%%%%%%%%%%%%%%%%%%

%----------------------------------------------------------------------------------------
%	PACKAGES AND OTHER DOCUMENT CONFIGURATIONS
%----------------------------------------------------------------------------------------

\documentclass[margin, 10pt]{res} % Use the res.cls style, the font size can be changed to 11pt or 12pt here

\usepackage{hyperref}

%\setlength{\textwidth}{5.1in} % Text width of the document
\setlength{\sectionwidth}{1.5in}
%\addtolength{\topmargin}{-0.5in}
%\addtolength{\textheight}{1.0in}

\begin{document}

%----------------------------------------------------------------------------------------
%	NAME AND ADDRESS SECTION
%----------------------------------------------------------------------------------------
\moveleft.5\hoffset\centerline{\large\bf Josh Bevan}
\moveleft.5\hoffset\centerline{jjbevan2@illinois.edu}
\moveleft.5\hoffset\centerline{(774) 329-0343 $\vert$ JoshBevan.com}
 \vskip 1.5em

%----------------------------------------------------------------------------------------

\begin{resume}
%----------------------------------------------------------------------------------------
%	EDUCATION SECTION
%----------------------------------------------------------------------------------------
\section{EDUCATION}
{\bf PhD, Computer Science [Scientific Computing group]} \\
	Univ. of Illinois Urbana-Champaign, Urbana, IL, Expected Graduation: 2021;\\
	Advisor: Prof. Andreas Kloeckner
	
{\bf MS in Mechanical Engineering [Thermofluids Concentration]} \\
	University of Massachusetts, Lowell, MA, 2015;\\
	\textit{Thesis:} ``Vortex Dominated Flows: A High-order, Conservative Eulerian Simulation Method." Advisor: Prof. D.J. Willis

{\bf Bachelor of Science in Mechanical Engineering}\\
	University of Massachusetts, Lowell, MA, 2013\\
	Senior Capstone: ``Autonomous Control of a Hovercraft."

%----------------------------------------------------------------------------------------
%	RESEARCH
%----------------------------------------------------------------------------------------
\section{RESEARCH \\EXPERIENCE}
{\bf The Scalability of ParSplice, Co-Design Summer School} \hfill May-August 2017\\
Los Alamos National Laboratory, Los Alamos, NM
\begin{itemize} \itemsep -2pt
\item Assessed current baseline performance of ParSplice for current architectures, and determined several avenues of improvement
\item Developed separate threaded instance for splicer task, removing serial bottleneck
\item Developed proof of concept high-temperature exploration routine for quick evaluation of potential landscape
\end{itemize}

{\bf Research Assistantship Appointment} \hfill 2016-2017\\
Advisor: Prof. Andreas Kloeckner, Urbana, IL
\begin{itemize} \itemsep -2pt
\item Investigated extension of Quadrature-by-Expansion (QBX) integral equation methods to volume-type integrals.
\item Worked on adapting a Fast Multipole Method (FMM) for unstructured simplicial meshes for finite element-like discretizations
\item Investigated importance of kernel induced effects on QBX approximations for volume potentials
\end{itemize}

{\bf Application of Discontinuous Galerkin and Vortex Transport\\ Methods to Turbine-Turbine Interaction Simulations} \hfill 2013-2015\\
Advisor: Prof. D.J. Willis, Lowell, MA
\begin{itemize} \itemsep -2pt
\item Implemented a high-order DG solver capable of arbitrary order solution of the Euler equations using Line-DG style approach.
\item Implemented high-order velocity-vorticity inviscid solver (2D domains) for calculation of velocity fields due to vorticity by means of Biot-Savart integral.
\item Investigated accuracy effects of integrable singularities on computed velocities for both singular and de-singularized kernels.
\item Performed validation of method and solver using test cases with known analytical solutions as well as empirical results from literature.
\end{itemize}

 \vskip -.5em

{\bf A Parallelizable Solver of Inviscid Fluid Flow} \hfill  2013\\
Special Topics Directed Study with Prof. J.P. Trelles, Lowell, MA
\begin{itemize} \itemsep -2pt
\item Developed a potential flow solver via solution of Poisson equation.
\item Implemented in C with the PETSc library to allow scalable parallelism via MPI.
\item Measured scaling efficiency for several test cases. Examined effect of mesh/node conditioning on inter-processor communication and effect on scaling.
\end{itemize}

  \vskip -.5em

{\bf Undergraduate Capstone: Autonomous Hovercraft Project} \hfill  2012 - 2013\\
Autonomous Robotic Systems Laboratory; Lowell, MA
\begin{itemize} \itemsep -2pt
\item Developed closed feedback two-stage linearized control algorithm for navigation.
\item Implemented real-time control program on micro-controller using IMU output and remote telemetry over serial radio.
\item Created autonomously navigating hovercraft platform with full servo/brushless motor control via micro-controller.
\end{itemize}

 \vskip -.5em

{\bf Farnsworth-Hirsch Fusor Project} \hfill  2008 - 2011\\
Van der Graff Accelerator Lab; Lowell, MA
\begin{itemize} \itemsep -2pt
\item Designed and built high vacuum pump systems and chambers.
\item Modified and built high-voltage and current-regulated power supplies.
\item Worked with RF and ECR ionization and plasma manipulation techniques.
\end{itemize}

%----------------------------------------------------------------------------------------
%	Teaching
%----------------------------------------------------------------------------------------
\section{TEACHING}
{\bf Teaching Assistant, CS 450:\,Numerical Analysis} \hfill Fall 2017\\
Univ. of Illinois Urbana-Champaign, Urbana, IL

{\bf Teaching Assistant, CS 450:\,Numerical Analysis} \hfill Spring 2017\\
Univ. of Illinois Urbana-Champaign, Urbana, IL

{\bf Introduction to Discontinuous Galerkin Methods} \hfill Spring 2015\\
University of Massachusetts, Lowell, MA
\begin{itemize} \itemsep -2pt
\item Developed curriculum for introductory graduate course on DG methods.
\item Created course materials for curriculum: slides, demonstrative programs.
\item Performed, recorded, and edited 44 video lectures series covering course material.
\item Video lectures available online at: \url{www.bit.ly/IntroDG}
\end{itemize}

{\bf Programming Language Proficiency}
\begin{itemize} \itemsep -2pt 
\item Proficient: MATLAB, Python
\item Competent: C/C++/Fortran
\item Also familiar with various scientific computing-related packages including OpenCL, PETSC, and the Python ecosystem of packages (e.g. NumPy, SciPy, etc.)
\end{itemize}
%----------------------------------------------------------------------------------------
%	PAPERS
%----------------------------------------------------------------------------------------
\section{PUBLICATIONS}

J.J. Bevan, D.J. Willis. "A High-Order Conservative Eulerian Simulation Method for Vortex Dominated Flows." 46th AIAA Fluid Dynamics Conference. 2016.

%----------------------------------------------------------------------------------------
\section{POSTERS}

 Ramakrishnaiah, Landsgesell, Zhou, Linck, Ramil, Bevan, Perez, Vernon, Swinburne, Pavel, Junghans.\\ ``Facilitating the Scalability of ParSplice for Exascale Testbeds.'' International Conference for High Performance Computing, Networking, Storage and Analysis. 2017.
%----------------------------------------------------------------------------------------
%	PATENTS
%----------------------------------------------------------------------------------------
\section{PATENTS} 

Sizer, C; Bevan, J; Olson K. 2012. Formulation and Processing of a High-Protein, Isotonic Beverage For the Treatment of Metabolic Disorders. Patent Pending.
 
 %----------------------------------------------------------------------------------------
%	EXTRA-CURRICULAR ACTIVITIES SECTION
%----------------------------------------------------------------------------------------

\section{AWARDS \& \\ACTIVITIES} 

Andrew \& Shana Laursen Fellowship, UIUC, 2016\\
Society for Industrial and Applied Mathematics (SIAM), UIUC Chapter Officer, 2016
Craig T. Douglas Undergraduate Research Award, UMass Lowell, 2013\\
Certified Associate in Project Management, Project Mgmt. Institute (PMI), 2014\\
Founder, Near Space High-Altitude Balloon Club, 2008-2012\\
Eagle Scout, Boy Scouts of America, 2004
 
%----------------------------------------------------------------------------------------
%	PROFESSIONAL EXPERIENCE SECTION
%----------------------------------------------------------------------------------------
\section{EMPLOYMENT}

{\bf Engineer /  International Project Manager} \hfill Dec. 2011 to Feb. 2016\\
Cambrooke Therapeutics Inc.; Ayer, MA
\begin{itemize} \itemsep -2pt
\item Designed, built, and optimized thermo-fluid processing systems.
\item Designed, built, and integrated custom hardware and software automation systems for aseptic clinical metabolics production conforming to FDA standards.
\item Implemented regulatory compliant products and procedures (e.g. 21 CFR 11 compliant electronic records system, 21 CFR 113/114 Thermal Processing, etc).
\item Managed projects including international market expansion, logistics, regulatory acceptance and compliance, import/export, supply chain, and international distributor management.
\item Responsible for new equipment/instrumentation installation and validation.
\end{itemize}

 \vskip -.5em

{\bf Engineer} \hfill Nov. 2010 to Dec. 2011\\
JSB Industries; Lawrence, MA
\begin{itemize} \itemsep -2pt
\item Troubleshooted and improved the safety, reliability, and efficiency of processing and packaging equipment and instrumentation.
\item Created data-driven, predictive preventative maintenance scheduling system.
\item Improved production throughput and uptime, minimized production waste.
\end{itemize}

\end{resume}
\end{document}