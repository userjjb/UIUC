\documentclass[letterpaper,10pt]{article}

\usepackage[english]{babel}
\usepackage[utf8]{inputenc}
\usepackage{amsmath}
\usepackage{graphicx}
\usepackage[colorinlistoftodos]{todonotes}
\usepackage[top=1in, bottom=1in, left=1in, right=1in]{geometry}
\usepackage[small]{titlesec}

\newcommand{\bes}{\begin{equation*}}
\newcommand{\ben}[1]{\begin{equation}\label{#1}}
\newcommand{\ees}{\end{equation*}}
\newcommand{\be}{\begin{equation}}
\newcommand{\ee}{\end{equation}}

\newcommand{\bm}[1]{% inline column vector
	\begin{bmatrix}#1\end{bmatrix}%
}

\begin{document}

\begin{flushright}
{\Large Josh Bevan - HW4 Q4 - CS450}
\end{flushright}
\vskip -0.1in
\hrule
\vskip 0.3in

\section*{At what point does $f$ attain a minimum?}
Gradient:
$$\nabla f(\mathbf{x}) = \begin{bmatrix}
    2x_1^3 - 2x_1x_2 + x_1 - 1 \\
    -x_1^2 + x_2 \end{bmatrix},$$
Hessian:
$$\mathbf{H}(\mathbf{x}) = \begin{bmatrix}
    6x_1^2 - 2x_2 + 1 & -2x_1 \\
    -2x_1 & 1 \end{bmatrix}.$$
There is a critical point at $\mathbf{x^*} = \begin{bmatrix} 1 & 1 \end{bmatrix}$ which is a
local minimum because $\mathbf{H}_f(\mathbf{x^*})$ is positive definite.

\section*{Perform one iteration of Newton's method for minimizing $f$ using starting point $x_0=\begin{bmatrix} 2 & 2 \end{bmatrix}^T$.}
Form linear system using starting point:

$$\begin{bmatrix} 21 & -4 \\ -4 & 1 \end{bmatrix}
    \begin{bmatrix} s_1 \\ s_2 \end{bmatrix} =
    \begin{bmatrix} -9 \\ 2 \end{bmatrix}$$
Solving yields
$\mathbf{s} = \begin{bmatrix} -0.2 & 1.2 \end{bmatrix}^T$, so next step's value is $\begin{bmatrix} 1.8 & 3.2 \end{bmatrix}^T$.

\section*{In what sense is this a good step? In what sense is this a bad step?}
The step is good insofar as it has reduced the value of the function $f(\mathbf{x}_1) = 0.32$, down from 2.5. However it's bad because we are actually \textit{farther} away from the correct solution moving from $\begin{bmatrix} 2 & 2 \end{bmatrix}^T$ to $\begin{bmatrix} 1.8 & 3.2 \end{bmatrix}^T$, away from the correct value of $\begin{bmatrix} 1 & 1 \end{bmatrix}^T$.
 
\end{document}

%\begin{figure}[!htb]
%\centering
%\includegraphics[width=0.6\textwidth]{Unrolled.PNG}
%\caption{\label{fig:unrolled}"Unrolled" ring, coincident nodes at either end.}
%\end{figure}

%\bes
%\alpha_j = (\bm{r\\r^*},\bm{r^*\\r}) / (\bm{0&A\\A^T&0}\bm{p^*\\p},\bm{p^*\\p}) \rightarrow (r,r^*)/(Ap,p^*)
%\ees