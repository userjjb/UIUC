\documentclass[letterpaper,10pt]{article}

\usepackage[english]{babel}
\usepackage[utf8]{inputenc}
\usepackage{amsmath}
\usepackage{graphicx}
\usepackage[colorinlistoftodos]{todonotes}
\usepackage[top=1in, bottom=1in, left=1in, right=1in]{geometry}
\usepackage[small]{titlesec}

\newcommand{\bes}{\begin{equation*}}
\newcommand{\ben}[1]{\begin{equation}\label{#1}}
\newcommand{\ees}{\end{equation*}}
\newcommand{\be}{\begin{equation}}
\newcommand{\ee}{\end{equation}}

\newcommand{\bm}[1]{% inline column vector
	\begin{bmatrix}#1\end{bmatrix}%
}

\begin{document}

\begin{flushright}
{\Large Josh Bevan - HW5 Q5 - CS450}
\end{flushright}
\vskip -0.1in
\hrule
\vskip 0.3in

\section*{ Use the method of undetermined coefficients to determine the nodes and weight for a three-point Chebyshev quadrature rule on the interval $[-1,1]$.}

A three point quadrature rule must satisfy the following fo r the method of undetermined coefficients:
$$\bm{ 1 & 1 &1 \\ x_1 & x_2 & x_3 \\ x_1^2 & x_2^2 & x_3^2}
    \begin{bmatrix} w_1 \\ w_2 \\ w_3 \end{bmatrix} =
    \begin{bmatrix} a-b \\ (a^2-b^2)/2 \\ (a^3-b^3)/3 \end{bmatrix}$$
    
For a constant set of weights $w_1 = w_2 = w_3$ the constant part must then satisfy:
$$w_1 +w_2 +w_3 = 3w = 1-(-1) \rightarrow w=2/3$$

We now have two equations and three unknowns to satisfy for $x_n$, so we add the additional criteria that the nodes will be symmetric in the interval; that is $x_1 = -x_3$. The two equations to satisfy are now:
$$2/3 \,(x_1 + x_2 - x_1) = 0 \rightarrow x_2 = 0$$
$$2/3 \,(x_1^2 + x_2^2 +x_1^2) = 2/3 \rightarrow 2/3 \,(2x_1^2 + 0^2) =2/3 \rightarrow x_1 = 0.5^{1/2} \approx 0.707$$

so $x = \{-0.707, 0, 0.707\}$, $w = \{2/3, 2/3, 2/3\}$

\section*{What is the degree of the resulting rule?}
We can determine the degree by seeing the highest order polynomial it exactly integrates. Consider a lower order test case:
$$ f(x) = 4x^3+4x^2 $$
$$ \int f(x) =  x^4 + 4/3 x^3 \rightarrow \int_{-1}^1 f(x) = 8/3 \approx 2.666666...$$
If we apply our quadrature rule, we get
$$ w \sum_i f(x_i) = 2.666666666666667$$
which agrees within machine precision of the exact answer.
Looking at the lowest order example polynomial we don't integrate exactly:
$$ f(x) = 5x^4+4x^2 $$
$$ \int f(x) =  x^5 + 4/3 x^3 \rightarrow \int_{-1}^1 f(x) = 14/3 \approx 4.666666...$$
If we apply our quadrature rule, we get
$$ w \sum_i f(x_i) = 4.3333333333333339$$
indicating that this quadrature is not exact for 4th order polynomials; therefore the degree of the resulting rule is 3rd order.


\end{document}