\documentclass[letterpaper,10pt]{article}

\usepackage[english]{babel}
\usepackage[utf8]{inputenc}
\usepackage{amsmath}
\usepackage{graphicx}
\usepackage[colorinlistoftodos]{todonotes}
\usepackage[top=1in, bottom=1in, left=1in, right=1in]{geometry}
\usepackage[small]{titlesec}

\newcommand{\bes}{\begin{equation*}}
\newcommand{\ben}[1]{\begin{equation}\label{#1}}
\newcommand{\ees}{\end{equation*}}
\newcommand{\be}{\begin{equation}}
\newcommand{\ee}{\end{equation}}

\newcommand{\bm}[1]{% inline column vector
	\begin{bmatrix}#1\end{bmatrix}%
}

\begin{document}

\begin{flushright}
{\Large Josh Bevan - HW5 Q2 - CS450}
\end{flushright}
\vskip -0.1in
\hrule
\vskip 0.3in

\textit{Given the three data points $(-1,1), (0,0), (1,1)$, determine the interpolating polynomial of degree two:}

\section*{Using the monomial basis:}
In general:
$$\bm{ 1 & t_1 &t_1^2 \\ 1 & t_2 & t_2^2 \\1 & t_3 & t_3^2}
    \begin{bmatrix} x_1 \\ x_2 \\ x_3 \end{bmatrix} =
    \begin{bmatrix} y_1\\y_2\\y_3 \end{bmatrix}$$
    with $p(t) = x_1 + x_2 t + x_3 t^2$
    
For our data:
$$\bm{ 1 & -1 &1 \\ 1 & 0 & 0 \\1 & 1 & 1}
    \begin{bmatrix} x_1 \\ x_2 \\ x_3 \end{bmatrix} =
    \begin{bmatrix} 1\\0\\1 \end{bmatrix}$$
    which yields $x = [0 \,  0 \, 1]^T$
    
    So our interpolating polynomial is simply $p(t) = t^2$.
    
    \section*{Using the Lagrange basis:}
    \section*{Using the Newton basis:}
    
    

\end{document}