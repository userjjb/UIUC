\documentclass[letterpaper,10pt]{article}

\usepackage[english]{babel}
\usepackage[utf8]{inputenc}
\usepackage{amsmath}
\usepackage{graphicx}
\usepackage[colorinlistoftodos]{todonotes}
\usepackage[top=1in, bottom=1in, left=1in, right=1in]{geometry}
\usepackage[small]{titlesec}

\newcommand{\bes}{\begin{equation*}}
\newcommand{\ben}[1]{\begin{equation}\label{#1}}
\newcommand{\ees}{\end{equation*}}
\newcommand{\be}{\begin{equation}}
\newcommand{\ee}{\end{equation}}

\begin{document}

\begin{flushright}
{\Large Josh Bevan - HW 4 Q1 - CS556}
\end{flushright}
\vskip -0.1in
\hrule
\vskip 0.3in

\hskip -.3in{\large \textit{Consider a general 6x6 tridiagonal nonsingular matrix.}}
\bes
  A =
\begin{bmatrix}
2&-1&0&0&0&0\\
-1&2&-1&0&0&0\\
0&-1&2&-1&0&0\\
0&0&-1&2&-1&0&\\
0&0&0&-1&2&-1\\
0&0&0&0&-1&2
\end{bmatrix}
\ees

\section*{What can be said about its ILU(0) factorization (when it exists)?}
By definition the zero pattern of L and U (from ILU(0)) will match that of A. More interesting to note though is that the L and U resulting from ILU(0) will be the same as the L and U resulting from a full LU factorization. This is a consequence of the tridiagonal nature of A; a full LU factorization of a tridiagonal matrix is confined to the diagonal and sub-diagonal for L and diagonal and super-diagonal for U. There is no fill-in that exists for the ILU(0) factorization of A to zero, all non-zero terms of the factorization already reside within the non-zero pattern of A.

\section*{Now apply the permuation $\pi=[1,3,5,2,4,6]$ to the matrix symmetrically (i.e., both rows and columns are permuted).}
\bes
  A_{\pi \pi} =
\begin{bmatrix}
2&0&0&-1&0&0\\
0&2&0&-1&-1&0\\
0&0&2&0&-1&-1\\
-1&-1&0&2&0&0&\\
0&-1&-1&0&2&0\\
0&0&-1&0&0&2
\end{bmatrix}
\ees

\section*{What is the sparsity pattern of the permuted matrix?}
See above. The symmetric permutation permutes the diagonal terms, but keeps them confined to the diagonal. The sub/super-diagonals have been moved elsewhere in the matrix, but the resultant matrix (and sparsity pattern) is still symmetric.

\section*{Show the locations of the fill-in elements in the ILU(0) factorization of the permuted matrix.}
The ILU(0) factorization has the same zero pattern as $A_{\pi\pi}$, the fill-in elements that would have resulted from a full LU factorization can be collected in a matrix R (here L and U are the ILU(0) factorizations):
\bes A=LU-R \ees
The fill-in elements of R in the example matrix A chosen are:
\bes
R=
\begin{bmatrix}
 0&  0&  0&  0&  0&0 \\ 
 0&  0&  0&  0&  0&0 \\ 
 0&  0&  0&  0&  0&0 \\ 
 0&  0&  0&  0&  -0.5&0 \\ 
 0&  0&  0&  -0.5&  0&-0.5 \\ 
 0&  0&  0&  0&  -0.5&0 
\end{bmatrix}
\ees
\end{document}

%\begin{figure}[!htb]
%\centering
%\includegraphics[width=0.6\textwidth]{Unrolled.PNG}
%\caption{\label{fig:unrolled}"Unrolled" ring, coincident nodes at either end.}
%\end{figure}