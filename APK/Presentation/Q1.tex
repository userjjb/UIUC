\documentclass[letterpaper,10pt]{article}

\usepackage[english]{babel}
\usepackage[utf8]{inputenc}
\usepackage{amsmath}
\usepackage{graphicx}
\usepackage[colorinlistoftodos]{todonotes}
\usepackage[top=0.5in, bottom=0.75in, left=0.9in, right=0.9in]{geometry}
\usepackage[small]{titlesec}

\newcommand{\bes}{\begin{equation*}}
\newcommand{\ben}[1]{\begin{equation}\label{#1}}
\newcommand{\ees}{\end{equation*}}
\newcommand{\be}{\begin{equation}}
\newcommand{\ee}{\end{equation}}

\titlespacing{\section}{0pt}{\parskip}{-\parskip}

\begin{document}

\begin{flushright}
{\Large Fourier Transform modified kernel - November}
\end{flushright}
\vskip -0.1in
\hrule
\vskip 0.4in

\vskip 0.1in
\section*{Take Fourier transform of $\frac{1}{\sqrt{r^2+a^2}}$:}

\begin{align*}
\int_{-\infty}^\infty\int_{-\infty}^\infty\int_{-\infty}^\infty \frac{1}{\sqrt{r^2+a^2}}e^{-2 \pi i\vec k\cdot \vec r }\,dx\,dy\,dz&=
\int_0^\infty\int_0^\pi \int_0^{2\pi} \frac{e^{-2 \pi ikr\cos(\theta)}}{\sqrt{r^2+a^2}}\,r^2\,\sin(\theta)\,d\phi\,d\theta\,dr\\\\
&=2\pi \int_0^\infty \int_0^\pi \frac{e^{-2 \pi ikr\cos(\theta)}}{\sqrt{r^2+a^2}}\,r^2\,\sin(\theta)\,d\theta\,dr\\\\
&=2\pi \int_0^\infty \frac{r^2}{\sqrt{r^2+a^2}} \int_0^\pi  e^{-2 \pi ikr\cos(\theta)}\sin(\theta)\,d\theta\,dr\,\\\\
&=2\pi \int_0^\infty \frac{r^2}{\sqrt{r^2+a^2}}\left(\frac{\sin(2\pi kr)}{\pi k r}\right)\,dr\\\\
&=\frac{2}{k} \int_0^\infty \frac{r}{\sqrt{r^2+a^2}}\,\sin(2\pi kr)\,dr
\end{align*}
We now find that the integral fails to converge, since $\frac{r}{\sqrt{r^2+a^2}} \to 1$ as $r \to \infty$, and so $\frac{r}{\sqrt{r^2+a^2}} \sin(2\pi kr)$ is undamped. However, we expect that a value for the integral exists since one exists for $\int_0^\infty sin(2 \pi kr) \,dr = 1/2 \pi k $, the ``physicist's point charge'' example. For this example they introduce a screening term $e^{-br}, b>0$ in the integral which artificially damps the integrand and forces it to converge, then they take the limit as $b\rightarrow 0$. Apply same principle here:

\begin{align*}
\frac{2}{k} \int_0^\infty \frac{r}{\sqrt{r^2+a^2}}\,\sin(2\pi kr)\,dr &= \lim_{b \to 0}\, \frac{2}{k} \int_0^\infty \frac{r}{\sqrt{r^2+a^2}}\,e^{-b r}\sin(2\pi kr)\,dr\\\\
&= \lim_{b \to 0}\, \frac{2}{k} \int_0^\infty \frac{r}{\sqrt{r^2+a^2}}\, \frac{( e^{(b - 2 \pi k i)r} - e^{-(b + 2 \pi k i)r})}{2 i}\, dr\\\\
&= \frac{2}{\pi k^2} + \frac{i \pi a}{2k} (Y_1(-i 2 \pi a k)-Y_1(i 2 \pi a k))\\\\
&= \frac{2}{\pi k^2} - \frac{\pi a}{k} \operatorname{Im}Y_1(i 2 \pi a k)
\end{align*}

If we realize that $\lim_{a \to 0} a Y_1(i 2 \pi a k) = i/\pi^2 k$, we see that the limit of the Fourier transform for $a \to 0$ is simply $1/\pi k^2$, and we recover the Fourier transform of $1/r$ (i.e $1/\sqrt{r^2 + a^2}$ for $a=0$).

\vskip 0.1in
\section*{An alternate form:}
While the previous solution seems to reduce to the expected case for $a=0$, Mathematica/Fourier transform tables are unable to verify that the inverse Fourier transform recovers $1/\sqrt{r^2 + a^2}$. I managed to find another solution form of the Fourier transform that is readily verifiable, however it requires a leap of faith at one step that could do with some additional rigor.

We could integrate by parts to try and get an integrand that is better behaved, with the troublesome term being $\frac{r}{\sqrt{r^2+a^2}}$.

$$
u = \frac{r}{\sqrt{r^2+a^2}} \;\; u' = \frac{a^2}{(r^2+a^2)^{3/2}}\\\\
$$
$$
v' = \sin{2 \pi k r} \;\; v = \frac{- \cos{2 \pi k r}}{2 \pi k}
$$
$$
\int_0^\infty uv' \,dr = uv\Big|_0^\infty - \int_0^\infty u' v \,dr
$$
\begin{align*}
\frac{2}{k} \int_0^\infty \frac{r}{\sqrt{r^2+a^2}}\,\sin(2\pi kr)\,dr &= \frac{2}{k}\left( \int_0^\infty \frac{a^2}{(r^2+a^2)^{3/2}}\, \frac{\cos{2 \pi k r}}{2 \pi k}\,dr - 
\frac{r}{\sqrt{r^2+a^2}} \, \frac{ \cos{2 \pi k r}}{2 \pi k} \Big|_0^\infty \right) \\\\
&=\frac{2}{k} \left( a K_1(2 \pi a k) - \lim_{r \to \infty}\left( \frac{r}{\sqrt{r^2+a^2}} \frac{\cos{2 \pi k r}}{2 \pi k} \right) \right) 
\end{align*}

The issue is the $\lim_{r \to \infty}\left( \frac{r}{\sqrt{r^2+a^2}} \cos{2 \pi k r}\right)$ which is in $[-1,1]$, but is otherwise unspecified. However, the limit  in $r \to \infty$ should be a value such that the resulting transform reduces to $1/\pi k^2$ when $a \to 0$, recovering the transform of $1/r$. If we notice that  $\lim_{a \to 0} a K_1(2 \pi a k) = 1/2 \pi k$, then we can see that $\lim_{r \to \infty}\left( \frac{r}{\sqrt{r^2+a^2}} \cos{2 \pi k r}\right)$ must be 0 for the transform to recover the desired behavior for $a \to 0$.

Indeed, if we now test the proposed transform by taking the inverse Fourier transform, we find:
$$
2\pi \int_0^\infty \frac{2a}{k} K_1(2 \pi a k) \left(\frac{\sin(2\pi kr)}{\pi k r}\right)\,dk = \frac{1}{\sqrt{r^2+a^2}}\\\\
$$


\end{document}