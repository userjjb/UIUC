\documentclass[letterpaper,10pt]{article}

\usepackage[english]{babel}
\usepackage[utf8]{inputenc}
\usepackage{amsmath}
\usepackage{graphicx}
\usepackage[colorinlistoftodos]{todonotes}
\usepackage[top=0.5in, bottom=0.75in, left=0.9in, right=0.9in]{geometry}
\usepackage[small]{titlesec}

\newcommand{\bes}{\begin{equation*}}
\newcommand{\ben}[1]{\begin{equation}\label{#1}}
\newcommand{\ees}{\end{equation*}}
\newcommand{\be}{\begin{equation}}
\newcommand{\ee}{\end{equation}}

\titlespacing{\section}{0pt}{\parskip}{-\parskip}

\begin{document}

\begin{flushright}
{\Large Josh Bevan - HW1Q3 - CS598APK}
\end{flushright}
\vskip -0.1in
\hrule
\vskip 0.4in

\vskip 0.1in
\section*{
Show that the combined
operator $G(F\phi)(x)$ can be recast in the same form as the above two,
i.e.
$$
(G\circ F) \phi(x):=\int_a^x k_3(x,y) \phi(y) \, d y.
$$}

Given two integral operators
$$  F \phi(z):=\int_a^z k_1(z,y) \phi(y) \, d y$$\\ $$ G \phi(x):=\int_a^x k_2(x,z) \phi(z) \, d z $$
 we can form the composition:
$$
(G\circ F) \phi(x) = \int_a^x k_2(x,z) 
\int_a^z k_1(z,y) \phi(y) d y \;
 d z.
$$
We are free to move $k_2(x,z)$ inside the inner integral since it is not a function of $z$.
$$
(G\circ F) \phi(x) = \int_a^x  
\int_a^z k_2(x,z) \, k_1(z,y) \phi(y) \, d y \;
 d z.
$$

We can then exchange the order of integration to obtain
$$
(G\circ F) \phi(x) = \int_a^z  
\int_y^x k_2(x,z) \, k_1(z,y) \phi(y) \, d z \;
 d y.
$$

We can then remove $\phi(z)$ from the inner integral since it does not depend on $y$
\ben{eq:a}
(G\circ F) \phi(x) = \int_a^z \phi(y) 
\int_y^x k_2(x,z) \, k_1(z,y) \, d z \;
 d y.
\ee

We can now see that the inner integral has the form of $f(x,y)$ thanks to the integration in $z$.

\vskip 0.1in
\section*{Give an expression for the kernel $k_3$.}
If we think of this inner integral as another possible kernel function
$$
k_3(x,y) = \int_y^x k_2(x,z) \, k_1(z,y) \, d z
$$

and make the appropriate substitution in (\ref{eq:a}) we obtain
$$
(G\circ F) \phi(x) = \int_a^x k_3(x,y) \phi(y) \; d y.
$$

\end{document}